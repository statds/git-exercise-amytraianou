\documentclass[12pt, letterpaper, titlepage]{article}
\usepackage[utf8]{inputenc}
\usepackage{amsmath}
\usepackage{booktabs}
\usepackage{amsthm}
\usepackage{xtable}
\usepackage{graphicx}
\usepackage[margin=1in]{geometry}
\usepackage{hyperref}
\hypersetup{colorlinks = true, linkcolor = blue, citecolor=blue, urlcolor = blue}
\usepackage{natbib}
\usepackage{enumitem}
\usepackage{setspace}

\usepackage[pagewise]{lineno}
%\linenumbers*[1]
% %% patches to make lineno work better with amsmath
\newcommand*\patchAmsMathEnvironmentForLineno[1]{%
 \expandafter\let\csname old#1\expandafter\endcsname\csname #1\endcsname
 \expandafter\let\csname oldend#1\expandafter\endcsname\csname end#1\endcsname
 \renewenvironment{#1}%
 {\linenomath\csname old#1\endcsname}%
 {\csname oldend#1\endcsname\endlinenomath}}%
\newcommand*\patchBothAmsMathEnvironmentsForLineno[1]{%
 \patchAmsMathEnvironmentForLineno{#1}%
 \patchAmsMathEnvironmentForLineno{#1*}}%

\AtBeginDocument{%
 \patchBothAmsMathEnvironmentsForLineno{equation}%
 \patchBothAmsMathEnvironmentsForLineno{align}%
 \patchBothAmsMathEnvironmentsForLineno{flalign}%
 \patchBothAmsMathEnvironmentsForLineno{alignat}%
 \patchBothAmsMathEnvironmentsForLineno{gather}%
 \patchBothAmsMathEnvironmentsForLineno{multline}%
}

% control floats
\renewcommand\floatpagefraction{.9}
\renewcommand\topfraction{.9}
\renewcommand\bottomfraction{.9}
\renewcommand\textfraction{.1}
\setcounter{totalnumber}{50}
\setcounter{topnumber}{50}
\setcounter{bottomnumber}{50}

\newcommand{\jy}[1]{\textcolor{blue}{JY: #1}}
\newcommand{\eds}[1]{\textcolor{red}{EDS: (#1)}}

\title{Stats Paper}
\author{Amy Traianou
  Department of Statistics, University of Connecticut\\
  }
\date{Sep 2022}

\begin{document}

\maketitle

\begin{abstract}
summary
\end{abstract}

\section*{Introduction}
\addcontentsline{toc}{section}{Introduction}
Intro

\section*{Data Description}
\addcontentsline{toc}{section}{Data Description}
data

\section*{Methods}
\addcontentsline{toc}{section}{Methods}
methods 
Suppose the rdius of a circle is $r$. Then its area is 
\begin{equation}
\label{eq:area}
\pi r^2.
\end{equation}

To reference an equation:
Equation~\eqref{eq:area} is about the area of a circle. 

An unnumbered equation looks like:
\[
f(x)=\frac{1}{\sqrt{2\pi}}\exp\left( - \frac{x}{2} \ right),
\]


\section*{Results}
\addcontentsline{toc}{section}{Results}
results


\section*{Discussion}
discuss

\section*{References}
references

\end{document}
